\documentclass[10pt, compress]{beamer}

\usetheme{m}

\usepackage{booktabs}
\usepackage[scale=2]{ccicons}
\usepackage{minted}
\usepackage{wrapfig}

\usepgfplotslibrary{dateplot}

\usemintedstyle{trac}

\title{A Review of Analysis into the Ethics of Artificial Intelligence}
\subtitle{\small{SEM6120 - Fundamentals of Intelligent Systems}}
%\date{\today}
\author{Connor Luke Goddard (clg11)}
\institute{Aberystwyth University}

\begin{document}

\maketitle

\begin{frame}[fragile]
  \frametitle{Introduction}
  
  \small {
  
    As machines continue to evolve in technical capability, the ideas, discussions and arguments relating to the ethics of advanced AI are becoming ever more complex, insightful and controversial.
    
    \vspace{10pt}
    
    \textbf{Leading researchers}: Limitations won't fall to technology, but philosophy \& \textbf{ethics}.
    
    \vspace{10pt}
    
    As both a local society and a global civilisation, we all need to be fully aware and understanding as to the \textbf{true} benefits and risks of developing the next level of AI.
    
    }

\end{frame}

\begin{frame}[fragile]
  \frametitle{Ethics - Overview}
  
  \small{
  
    In its simplest form, ethics defines a system that enables us as humans to describe, discuss and categorise the behaviour we exhibit as either morally ‘right’ or ‘wrong’.
    
    \vspace{10pt}
    
    Ethics concerns itself exclusively with \textbf{human} behaviour - \textbf{sentience} vs \textbf{sapience}.
    
    \vspace{10pt}
    
    While all animals are able to appreciate `\textbf{conscious experiences}', humans are the only species possessing a `higher' form of intelligence.
    
    }

\end{frame}

%\begin{frame}[fragile]
%  \frametitle{Human Intelligence}
%  
%  \small{
%  
%    The results of human’s advanced cognitive abilities have defined the world we live in today. 
%    
%    Through lending our enhanced intelligence to complex problem solving, reasoning and communication; we have not only guaranteed our own position at the top of the food chain\ldots
%    
%    \ldots but now also seek to protect the interests and survival of \textbf{other species} 
%    
%    \vspace{5pt}
%    
%     - \small{``\textit{Then God said, Let us make mankind in our image, in our likeness, so that they may rule over the fish in the sea and the birds in the sky, over the livestock and all the wild animals, and over all the creatures that move along the ground.}" (Genesis 1:26 - New International Edition)}
%     
%     
%     }
%
%\end{frame}

\begin{frame}[fragile]
  \frametitle{Human Intelligence}
  
   \small{ 
   
   As a consequence of our enhanced `self-awareness', humans have throughout history sought to develop and refine better approaches to undertaking laborious or difficult tasks.
    
    Technology has typically provided the answer:
    
    \begin{itemize}
    	\item \textbf{Prehistoric Age}: Development of simple tools from natural resources and the discovery of fire
    	\item \textbf{1886}: Invention of the first `automobile' by Karl Benz
    	\item \textbf{1940}: Development of `The Bombe' by Alan Turing \& Gordon Welchman
    \end{itemize}
    
    Computer technology has developed at an exceptional rate over the last half-century. - The relationship between humans and computers has adjusted with equal pace.
    
%    Human desire for greater understanding and better communication + computing devices become both more powerful and accessible = 

Not only penetrates almost every aspect of human life, but has now also shown the potential for \textbf{re-shaping} our behaviour. 
     
     }

\end{frame}

%\begin{frame}[fragile]
%  \frametitle{Current State of AI}
%  
%   \small{ 
%   
%   The present role of AI technology has generally failed to conform to the public’s typical perception of what “AI” represents - AI algorithms have yet to begin demonstrating `true' \textbf{cognitive} behaviour.
%   
%   % Yet to present TRUE COGNITIVE behaviour.
%    
%    Despite this, AI now plays an enormous, yet typically hidden role in today's society:
%    
%    \begin{itemize}
%    	\item Understanding of customer behaviour (Big data)
%    	\item Expert systems (medical \& system diagnosis)
%    	\item Automated customer service solutions
%    \end{itemize}
%    
%    Until now, current AI solutions have not presented many unique ethical challenges: \textit{``\ldots designing a robot arm to avoid crushing stray humans is actually no more morally fraught than designing a flame-retardant sofa."}
%     
%     }
%
%\end{frame}

\begin{frame}[fragile]
  \frametitle{Ethical Analysis of Future Advanced AI}
  
   \small{ 
   
  	As AI move ever closer towards becoming true `thinking' machines, attention begins to turn towards what impacts such technology may have on future generations.
  	
  	Investigation focussed on analysis from three highly-cited pieces of literature:
  	
  	\begin{itemize}
    	\item ``The Ethics of Advanced Artificial Intelligence",  Bostrom \& Yudkowsky (2011)
    	\item ``Comment: Ethics of Artificial Intelligence" (Article), Nature (2015)
    	\item ``AI, Robotics \& The Future of Jobs", Pew Research Centre (2015)
    \end{itemize}
    
    With input from a collection of related material from within the AI ethical arena. 
     
     }

\end{frame}

%\begin{frame}[fragile]
%  \frametitle{Ethical Analysis of Future Advanced AI}
%  
%   \small{ 
%   
%  	`High-level' field of AI ethics typically decomposed into a collection of focus areas:
%  	
%  	\begin{enumerate}
%  		\item Employment, Economic \& Socialistic Issues
%  		\item Militarised AI \& Automated Weapons
%  		\item Domain-Specific vs. General Artificial Intelligence
%  		\item Super-intelligence \& Moral Status of Machines
%  	\end{enumerate}
%  	
%  	Many papers grouped discussion around these topics, as they provided a means of better relating ideas and arguments to real-world examples, while also highlighting the unique ethical challenges each area presented.
%     }
%
%\end{frame}

\begin{frame}[fragile]
  \frametitle{Employment, Economic \& Socialistic Issues}
  
   \small{ 
   
   	\textbf{Literature:} ``AI, Robotics \& The Future of Jobs", Pew Research Centre (2015)
   	
   	\vspace{10pt}
   	
   	Canvassing responses to the question:
  	
 	``\textit{Will automated AI systems and robotic devices have displaced more jobs than they have created by 2025?}"
   	
   	\vspace{5pt}
   	
 	\textbf{Themes:}
  	
  	\begin{enumerate}
  		\item Predicting the balance of jobs: man vs. machine
  		\item Future guarantee of jobs for humans
  		\item ETA for intelligent systems
  		\item Potential for income inequality and social imbalance
  	\end{enumerate}
   
%  	Concerns surrounding employment, and the subsequent impacts on future economic and social landscapes centre at some of the fiercest debates currently underway within the AI and business communities.
 
     }

\end{frame}

\begin{frame}[fragile]
  \frametitle{Militarised AI \& Automated Weapons}
  
   \small{ 
   
   	\textbf{Literature:} ``Comment: Ethics of Artificial Intelligence" (Article), Stuart Russell, Nature (2015)
   	
   	\vspace{10pt}
   	
   	Historically, military organisations have always been positioned at the forefront of technological advancement. - With AI, this is no different.
   	
%   	Idea of machines taking on decisions over life and death do not sit well with many people.
   	
   		       \begin{wrapfigure}{r}{0.40\textwidth}
  \begin{center}
  \vspace{-15pt}
    \includegraphics[width=0.35\textwidth]{code_darpa.jpg}
    \vspace{-20pt}
  \end{center}
  \end{wrapfigure}
   	
   	Author argues that all the `components' needed for LAWS already exist - ``\textit{they just need to be combined}" (FLA and CODE DARPA projects).
   	
   Mapping jurisdiction of existing humanitarian laws to the actions of autonomous weapons. - Gaps in applicability due to subjective reasoning.
   
   Need for the AI community to take a position on acceptable use of their technology.
   
     }

\end{frame}

\begin{frame}[fragile]
  \frametitle{Domain-Specific vs. General Artificial Intelligence}
  
   \small{ 
   
   \vspace{-10pt}
   
   	\textbf{Literature:} ``The Ethics of Advanced Artificial Intelligence", Bostrom \& Yudkowsky (2011)
   	
   	\vspace{10pt}
   	
   	AI has proven to be capable of surpassing human capability within a number of domains (e.g. `Deep Blue' defeating world chess champion Garry Kasparov (1997)).
   	
   	      
      \begin{wrapfigure}{r}{0.3\textwidth}
   \vspace{-15pt}
  \begin{center}
    \includegraphics[width=0.3\textwidth]{deepblue.jpg}
    \vspace{-30pt}
  \end{center}
  \end{wrapfigure}
   	
%   	While demonstrating exceptional insight within their specialist domain, these systems are restricted to only a very \textbf{specific} `knowledge-space'.
   	
   	\vspace{10pt}
   
   Future for AI will be to apply ``\textit{new or existing knowledge to multiple domains.}" - `Artificial \textbf{General} Intelligence'.
   
   \begin{itemize}
   	\item \textbf{Issue:} Requirement to conform to more than one set of ethical standards.
   	\item \textbf{Solution:} Extrapolation of the \textit{\textbf{distant consequences}} for behaviour.
   \end{itemize}
   
     }

\end{frame}

\begin{frame}[fragile]
  \frametitle{Super-intelligence}
  
   \small{ 
   
   \textbf{Literature:} ``The Ethics of Advanced Artificial Intelligence", Bostrom \& Yudkowsky (2011)
   
   \vspace{10pt}
   
     Machines intellectually-superior to humans present great potential benefits\ldots but equally large risks.
          
     \begin{itemize}
   	\item Capable of addressing all or nearly all of the main existential risks threatening human survival.  
   	\item Recognised as \textit{one of these existential risks} alongside detrimental environmental change and global disease pandemics.
   \end{itemize}
   
   \vspace{10pt}
   
   Creating a super-intelligent AI presents ``\textit{perhaps the ultimate challenge of machine ethics}":
      
   \begin{itemize}
   	\item[] Developing an artificial mind capable of becoming \textit{more ethical} than those who created it.
   \end{itemize}
  
   
     }

\end{frame}

\begin{frame}[fragile]
  \frametitle{Moral Status of Machines}
  
   \small{ 
   
   \textbf{Literature:} ``The Ethics of Advanced Artificial Intelligence", Bostrom \& Yudkowsky (2011)
   
   \vspace{10pt}
   
     The notion of super-intelligence raises for the first time, the question of when an AI stops being regarded as just an object, and instead begins to hold some form of \textbf{moral status}. 

	 If an AI demonstrated true `human-like' behaviour, could it ever be recognised as being \textbf{sapient}, without also possessing \textbf{sentience}?
	 
	 	            		       \begin{wrapfigure}{r}{0.3\textwidth}
  \begin{center}
    \includegraphics[width=0.3\textwidth]{coffee.jpg}
    \vspace{-20pt}
  \end{center}
  \end{wrapfigure}
	 
	 Would a non-sentient being hold the same moral status as a sentient person? 
	 
	 Unique issues relating to:
	 
	\begin{itemize}
   	\item Subjective vs. objective rate of time
   	\item Whole brain emulation / ``Uploading"
   \end{itemize}
	 
     }

\end{frame}

%\plain{Critique of Research Literature}

\begin{frame}[fragile]
  \frametitle{Critique - ``AI, Robotics \& The Future of Jobs"}
  
   \small{ 
   	
   	\textbf{Overview}: Provides insight into the current state of attitudes and opinions towards the expected impacts of AI on future jobs, sourced from responses submitted by `trustworthy' individuals, identified as such through their position and reputation within the AI and business communities.
   	
%   	Condensing of responses into a finite list of key themes provides reader with an immediate impression of the main arguments presented. 

		       \begin{wrapfigure}{r}{0.30\textwidth}
   \vspace{-25pt}
  \begin{center}
    \includegraphics[width=0.25\textwidth]{paper1.png}
  \end{center}
  \vspace{-50pt}
  \end{wrapfigure}
   	
   	Further statistical analysis of qualitative data may have helped to identify patterns behind motivations for the answers given by respondents 
%   	
%   	(e.g. links between positive/negative responses and respondent employer).
   	
%   	Survey is not representative - pre-selected respondents.
   	
   	Percentages have been used to disguise small sample sizes.
   	
   	Related research paper proposes highly-contrasting results.
   	
   	\footnotesize {
   	\begin{itemize}
   	\item[] Clear bias in favour of AI (publisher develops AI software)
   	\item[] Fails to publish explanation of audience size or methods for data collection.
   \end{itemize}
   
   }
   	
     }

\end{frame}


%\begin{frame}{Results: Example 2}
%
%\begin{columns}
%\begin{column}{.48\textwidth}
%
%\textbf{Result:}
%\begin{figure}[ht!]
%\centering
%\includegraphics[scale=0.22]{wiltshire_outside_10cm.png}
%\end{figure}
%
%\end{column}%
%\end{columns}
%
%\end{frame}



\plain{Any Questions? \\ \vspace{0.2cm} \footnotesize{Slide Design: Matthias Vogelgesang - (\href{https://github.com/matze/mtheme}{Github})}}

\end{document}
